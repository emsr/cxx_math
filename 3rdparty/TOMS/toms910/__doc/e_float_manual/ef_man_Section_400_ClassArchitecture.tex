The class architecture of the \efloat\ system has been described in
Section~\ref{sec:architecture}. The architecture is based on three main
components: the \efloatbaseclass\ class, the \efloatclass\ class and the
\efloatclass\ global arithmetic interface. The synopsis of these components is
described in the following sections.

\subsection{The \efloatbasehyperref\ Class}\label{subsec:efbaseclass}

The \efloatbaseclass\ class is a partly abstract base class from which the
\efloatclass\ class is derived. The $33$ public pure--virtual interface functions
of the \efloatbaseclass\ class describe all mathematical primitives needed
for real--numbered arithmetic.
The synopsis of the public interface to the \efloatbaseclass\ class is shown below.

\vspace{4.0pt}

\lstsetCPlusPlus
\lstinputlisting{CodeSample_EfloatBaseSynopsis.cpp}

\vspace{4.0pt}

\subsection{The \efloathyperref\ Class}\label{subsec:efclass}

The \efloatclass\ class implements all of the pure--virtual functions
of its base class and adds additional numerical constants and local
private utility functions.
The synopsis of the public interface to the \efloatclass\ class is shown below.

\vspace{4.0pt}

\lstsetCPlusPlus
\lstinputlisting{CodeSample_EfloatSynopsis.cpp}

\vspace{4.0pt}

\subsection{The \efloathyperref\ Global Arithmetic Interface}

The \efloatclass\ global arithmetic interface defines global functions
for arithmetic.
The synopsis of the \efloatclass\ global arithmetic interface is shown below.

\vspace{4.0pt}

\lstsetCPlusPlus
\lstinputlisting{CodeSample_EfloatGlobalMathSynopsis.cpp}

\vspace{4.0pt}

\subsection{Numeric Limits of the \efloathyperref\ Class}

In order to maintain consistency with C++ programming styles and idioms,
a template specialization of STL's standard numeric limits class has
been implemented. It is called
{\courier std::\-nu\-me\-ric\underline\ lim\-its<e\underline\ float>}.
The class synopsis is shown below.

\vspace{4.0pt}

\lstsetCPlusPlus
\lstinputlisting{CodeSample_EfloatNumLimSynopsis.cpp}

\vspace{4.0pt}

\subsection{Adapting Another MP Implementation for use with \efloathyperref}

From a software--architectural standpoint, it is quite straightforward
to adapt another MP implementation for use with \efloat.
In order to be used with \efloat, the MP needs to be implemented in a
class which is called \efloatclass. This adapted MP class must be
derived from the \efloatbaseclass\ class and it must also reside
within a unique namespace. Furthermore, the adapted MP class must fully
implement all of the $33$ pure--virtual functions of \efloatbaseclass,
which are listed in Section~\ref{subsec:efbaseclass}.
Although architecturally simple enough, the actual implementation details
of the functions needed for the
virtual interface --- such as fast multiplication --- can be
technically very challenging to implement in an efficient fashion.

Any optional number of the virtual functions needed for
the ``{\emph has--its--own}'' mechanism should also be implemented such
that they return {\courier true} if the MP's own function should
be used. Be sure to also implement the corresponding static functions.

When the adapted MP type is fully implemented, it can be selected
with a compile--time definition and used as a drop--in replacement
for the selected MP type in \efloat.

\subsection{The Complex Class Interface}

The \efcomplexclass\ class defines the \efloat\ interface to
complex--numbered values. The synopsis of the public interface
to the \efcomplexclass\ class is shown below.

\vspace{4.0pt}

\lstsetCPlusPlus
\lstinputlisting{CodeSample_EfloatComplexSynopsis.cpp}

\vspace{4.0pt}

