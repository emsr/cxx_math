\documentclass[11pt]{article}

\topmargin 0.3cm
\oddsidemargin 0cm
\textwidth 16cm
\textheight 21cm
\pagestyle{plain}
\frenchspacing{}

\jot 0.55cm
\renewcommand{\arraystretch}{1.4}
\def\bbox#1{\mbox{\boldmath$#1$}}
\def\eq{\;\;\Longleftrightarrow\;\;}
\def\trunc#1{[\mkern - 2.5 mu  [#1] \mkern - 2.5 mu ]}
\def\corresponds{{\lower.2ex\hbox{=}}{\rm\kern-.75em^\triangle}}
\def\succsim{\succ\kern-.9em_\sim\kern.3em}
\def\precsim{\prec\kern-1em_\sim\kern.3em}
\def\slantfrac#1#2{\kern1em^{#1}\kern-.3em/\kern-.1em_{#2}}
\def\lfrac#1#2{{}^{#1\!}\kern-.0em/_{#2}}

\setlength{\parindent}{0mm}
\setlength{\parskip}{1.0ex plus0.3ex minus0.5ex}

\sloppy{}

\usepackage{amsfonts}

\begin{document}

\begin{center}
\begin{tabular}{c}
\hline
\rule[-5mm]{0mm}{15mm} {\Huge \sf {\tt lerchphi} User's Guide} \\
\hline
\end{tabular}
\end{center}
%
\vspace{0.2cm}
%
\begin{center}
Sergej V. Aksenov$^{1,2)}$ and Ulrich D. Jentschura$^{3)}$
\end{center}
%
\vspace{0.2cm}
%
\begin{center}
$^{1)}$ Department of Microbiology and Immunology, \\
University of Michigan, Ann Arbor, MI 48109, USA \\[2ex]
$^{2)}$ Current address: Department of Zoology, \\
University of Cambridge, Downing Street, Cambridge, CB2 3EJ, UK \\[2ex]
$^{3)}$ Institute of Theoretical Physics, \\
Dresden University of Technology, 01062 Dresden, Germany \\[1ex]
{\bf Email:} sa354@cam.ac.uk, jentschura@physik.tu-dresden.de
\end{center}
%
\begin{center}
May 1, 2002
\end{center}
%
\vspace{0.1cm}
%
\begin{center}
\begin{minipage}{11.8cm}
{\underline{Abstract}}
This document describes the usage of a C program that calculates 
Lerch's $\Phi$ transcendent of real arguments using convergence acceleration 
methods (nonlinear sequence transformations and the combined 
nonlinear-condensation transformation).
\end{minipage}
\end{center}

\tableofcontents

\newpage

%
% Conditions of Use
%
\section{Conditions of Use}

\begin{itemize}
%
\item This document and software (file {\tt lerchphi.c}) described herein 
is copyright by Sergej V. Aksenov and Ulrich D. Jentschura.
%
\item This software and its constituent parts come with no warranty, 
whether expressed or implied, that it is free of errors or suitable 
for any specific purpose.  It must not be used to solve any problem, 
whose incorrect solution could result in injury to a person, institution, 
or property.  It is user's own risk to use this software or its parts and 
the authors disclaim all liability for such use.
%
\item This software is distributed ``as is''. In particular, no maintenance, 
support, troubleshooting, or subsequent upgrade is implied.
%
\item The use of this software must be acknowledged in any publication 
that contains results obtained with it; please cite~\cite{JeBeSoAkSaMo2002}.
%
\item The free use of this software and its parts is restricted for research 
purposes; commercial use require permission and licensing from Sergej V. 
Aksenov and Ulrich D. Jentschura.
\end{itemize}

\section{Availability}

The source of the LerchPhi program {\tt lerchphi.c}, the example driver file 
{\tt test.c}, example makefile, and a sample output of a run of 
{\tt test.c}, as well as this documentation can be downloaded from the
web pages of the authors~\cite{AkJeURL}.

\section{Purpose}

The C program contained in the file {\tt lerchphi.c} calculates 
Lerch's transcendent which is defined by the following infinite 
series:
%
\begin{equation}
\label{lerch}
\Phi(z,s,v) = \sum_{n=0}^\infty \frac{z^n}{(n+v)^s}\,,  \qquad
|z|<1\,, \qquad v \neq 0,-1,\dots\,.
\end{equation}
%
[See Eq.~(1) on p.~27 of Ref.~\cite{Ba1953vol1}].
We consider the case of real parameters $z$, $s$, and $v$ only.

The C language implementation of the combined nonlinear-condensation 
transformation (CNCT) described here is motivated by the virtual absence 
of freely available and transportable C code for Lerch's transcendent 
for real arguments. 

%
% Basic formulas
%
\section{Basic formulas}

Assume that the elements of
the sequence $\{ s_n \}_{n=0}^{\infty}$
represent partial sums 
%
\begin{equation}
\label{defsn}
s_n = \sum_{k=0}^{n} a(k)
\end{equation}
%
of an infinite series
%
\begin{equation}
\label{infser}
\sum_{k=0}^{\infty} a(k)\,,
\quad a(k) \geq 0\,,
\end{equation}
%
e.g. that in Eq.~(\ref{lerch}). We denote 
the limit $s \equiv \sum_{k=0}^{\infty} a(k)$. 

The first step of the CNCT is the Van Wijngaarden 
transformation~\cite{vW1965} of the
nonalternating input series~(\ref{infser}), whose partial 
sums are given by (\ref{defsn}), into an alternating
series
%
\begin{equation}
\label{vwijn}
s = \sum_{j=0}^{\infty} (-1)^j {\bf A}_j\,.
\end{equation}
%
The quantities ${\bf A}_j$ are defined according to 
%
\begin{equation}
\label{A2B}
{\bf A}_j \; = \; \sum_{k=0}^{\infty} \, {\bf b}_{k}^{(j)} \, ,
\end{equation}
%
where
%
\begin{equation}
\label{B2a}
{\bf b}_{k}^{(j)} \; = \; 2^k \, a(2^k\,(j+1)-1) \, .
\end{equation}

The second step is to apply an appropriate sequence transformation 
to the series of partial sums
%
\begin{equation}
\label{PSumS}
{\bf S}_n \; = \; \sum_{j=0}^{n} \, (-1)^j \, {\bf A}_j
\end{equation}
%
of the Van Wijngaarden transformed series~(\ref{vwijn}) of the first step.  
We use the delta sequence transformation~\cite{Si1979,We1989} which is 
constructed according to [Eq.~(8.4-4) of~\cite{We1989}]
%
\begin{equation}
\label{defdelta}
\delta_{k}^{(0)} \left( 1, {\bf S}_n \right) = \frac
{\displaystyle \sum_{j=0}^{k} \; ( - 1)^{j} \; 
{k \choose j} \; 
\frac {(j+1 )_{k-1}} {(k+1 )_{k-1}} \;
\frac {{\bf S}_{j}} {\omega_{j}} }
{\displaystyle \sum_{j=0}^{k} \; ( - 1)^{j} \; 
{k \choose j} \; 
\frac {(j+1 )_{k-1}} {(k+1 )_{k-1}} \;
\frac {1} {\omega_{j}} } \, .
\end{equation}
%
The quantities appearing in Eq.~(\ref{defdelta})
have the following interpretation: ${\bf S}_n$ are the elements of an 
alternating sequence whose convergence is to be accelerated, and the
$\omega_n$ are remainder estimates, i.e.~estimates for the
truncation error $\omega_n \approx r_n$ defined by $r_n = {\bf S}_n - s$.  
Remainder estimates are calculated as $\omega_n = {\bf S}_{n+1} - {\bf S}_n$.

%
% Usage of the program 
%
\section{Usage of the {\tt lerchphi} Program}

The program consists in a self-contained file {\tt lerchphi.c} written in 
ISO C. The file {\tt lerchphi.c} should be compiled with any user written 
main program.  A test program {\tt test.c} is provided which calculates 
Lerch's transcendent for several combinations of parameters, as well as a 
makefile (written for Unix systems) to build the executable {\tt test}. 
We built and ran an executable from this 
makefile on a Macintosh computer with the PowerPC G3 processor, 
Mac OS X v10.1.3, Darwin Kernel v5.3, and on an INTEL--based
LINUX system running SuSE Linux 7.3, as well as a Sun workstation
running SunOS. However, one should rather view these 
makefiles as templates and modify for the particular operating system and 
compiler.  

Further useful explanations are provided in the {\tt README} file
which forms part of the current distribution.

The main program should call {\tt lerchphi()} as follows
%
\begin{quote}
\fbox{{\tt flag = {\bf lerchphi}(\&z, \&s, \&v, \&acc, \&result, \&iter);}}
\end{quote}
%
Here, the variables should be defined in the main program and 
have the following types:
%
\begin{quote}
\fbox{{\tt {\bf int} flag, iter;}}
\end{quote}
%
\begin{quote}
\fbox{{\tt {\bf double} z, s, v, acc, result;}}
\end{quote}
%
The meaning of these variables is as follows: {\tt z}, {\tt s}, {\tt v} carry
the arguments of Lerch's transcendent $\Phi(z,s,v)$;
{\tt acc} stores the value of
the desired relative accuracy of the result; 
{\tt result} holds the calculated
value of Lerch's transcendent; and {\tt iter} is the 
number of iterations in the
convergence acceleration loop needed to reach 
the accuracy {\tt acc}. A positive
(nonzero) value of {\tt flag} signifies that 
an internal error occurred in the
calculation. 
The error flags are explained in Table~\ref{cerr} below.

The range of valid arguments $z$, $s$ and $v$
of the C program is determined by the properties of the defining series
representation (\ref{lerch}): we require 
$-1 < z < 1$, and $s,v$ to be real. The case
of negative $v$ is problematic. The C program stops with an error flag
if $v$ is a negative integer [one of the terms of the defining series
of Lerch's transcendent 
has a vanishing denominator for negative integer $v$]. 
Finally, if $v$ is negative non-integer, then we require
$s$ to be integer (for negative non-integer $v$ and non-integer $s$,
the C function ${\tt pow}$ is not well-defined).

%
% Details of the implementation
%
\section{Details of the Implementation}

The use of the CNCT for positive $z$, $z$ close to unity, removes
the principal numerical difficulties associated with the
slow convergence of the series (\ref{lerch}) in this parameter region.
For negative $z$, the partial sums of the -- in this case
alternating -- series (\ref{lerch}) are directly used as
input data for the delta transformation (\ref{defdelta}),
and the Van Wijngaarden step (\ref{vwijn}) -- (\ref{B2a})
of the CNCT may be skipped. This corresponds to the
standard use of nonlinear sequence transformations as efficient
accelerators for alternating series~\cite{We1989}.

The LerchPhi program has two subroutines: {\tt aj()} which calculates the Van 
Wijngaarden terms ${\bf A}_j$ in Eq.~(\ref{A2B}), and {\tt lerchphi()} that 
calculates the sequence of CNC transforms~(\ref{defdelta}).
Table~\ref{cerr} summarizes certain special conditions and genuine
error conditions handled by  {\tt lerchphi()} and lists the flag 
values generated.

\begin{table}[htb!]
\begin{center}
\begin{minipage}{16cm}
\begin{center}
%
\begin{tabular}{lll}
\hline
Special case & Program action & Flag \\
\hline
$z=0$ & Return $1/v^s$$^{\dagger}$ & 0\\
%
$z<-0.5$ & Use (\ref{defsn}) as input for (\ref{defdelta})$^{\ddagger}$ & 0 \\
%
$|z| \leq 0.5$ & Use term-by-term summation of (\ref{lerch}) & 0\\
%
$v<0\,,\,v \not \in \mathbb{Z} \setminus \mathbb{N}$, $s \in \mathbb{Z}$ 
& 
Use (\ref{lerchrel}) with $m=-\trunc{v}$ \hspace*{5mm} & 0 \\
%
$\vert z \vert \geq 1$ & 
Error exit & 1 \\
%
$v<0\,,v\in \mathbb{Z} \setminus \mathbb{N}$ & 
Error exit & 2 \\
%
$v<0\,,\,v \not \in \mathbb{Z} \setminus \mathbb{N}$, $s \not \in 
\mathbb{Z}$ & 
Error exit & 3 \\
%
Overflow in index of $a(k)$~[Eq.~(\ref{B2a})] \hspace*{5mm} & 
Return current iterate & 4 \\
%
Underflow in $\omega_{j}$~[Eq.~(\ref{defdelta})] & 
Return current iterate & 5 \\
%
Over max. iterations & 
Return current iterate & 6 \\
%
\hline
\multicolumn{3}{l}{$^{\dagger}$This is just the first term of the series 
  (\ref{lerch})}\\
\multicolumn{3}{l}{\hspace*{3mm} if we use the definition $0^0=1$.} \\[1ex]
\multicolumn{3}{l}{$^{\ddagger}$The Van Wijngaarden step (\ref{vwijn}) -- (\ref{B2a})
  is not needed because} \\
\multicolumn{3}{l}{\hspace*{3mm} the series is already alternating 
[replace ${\bf S}_n \to s_n$ in Eq.~(\ref{defdelta})].}\\
\hline
\end{tabular}
%
\caption{Special cases and error conditions in the C program 
{\tt lerchphi()} which calculates Lerch's transcendent.}
\label{cerr}
%
\end{center}
\end{minipage}
\end{center}
%
\end{table}

The program checks for parameters $z$, $s$, and $v$ to be in the domain 
of validity of the series representation of Lerch's transcendent~(\ref{lerch}). 
The program also refuses to calculate if $v<0$ and $s$ is not an integer, 
which is problematic for the standard C function {\tt pow()}.  
Note that the program can determine only up to machine accuracy
whether the parameters $v$ and $s$ are integer.

%
% Algorithms Used
%
\section{Algorithms Used}

We use the following relation [Eq.~(2) on p.~27 of Ref.~\cite{Ba1953vol1}]
%
\begin{equation}
\label{lerchrel}
\Phi(z,s,v) = z^m \, \Phi(z,s,m+v) + 
  \sum_{n=0}^{m-1} \frac{z^n}{(n+v)^s}\,.
\end{equation}
%
to transform from negative to positive values of $v$.

In the term-by-term summation of the defining series~(\ref{lerch}) for 
$|z| \leq 0.5$, we use the following recursion relationship for the 
terms $a(k)$ of Lerch's power series:
%
\begin{equation}
\label{termsrec}
\frac{a(k+1)}{a(k)} = z \left( \frac{v+k}{v+k+1} \right)^s\,.
\end{equation}

For the calculation of the Van Wijngaarden 
condensed series~(\ref{A2B}) of odd index,
we use the following recursion~\cite{JeBeSoAkSaMo2002,Da1969}:
%
\begin{equation}
\label{compute}
{\bf A}_{i+1} = 1/2\,({\bf A}_{i/2} - a(i/2))\,,
\end{equation}
%
where $i$ in the main loop of the program is an even integer.

The final output of the CNC transformation is a sequence
of approximants ${\cal T}_n \equiv {\cal T}_{\rm CNC}(n)$ that converge
to the value of $\Phi(z,s,v)$. We define the ratio of two
consecutive differences of approximants as
%
\begin{equation}
\label{xx}
x_n = \left| \frac{{\cal T}_{n} - {\cal T}_{n-1}}
  {{\cal T}_{n-1} - {\cal T}_{n-2}} \right|\,.
\end{equation}
%
The $x_n$ behave asymptotically, as shown in~\cite{JeBeSoAkSaMo2002} for a 
class of model problems, like 
a geometric series (within the region of convergence). 
Therefore, a good estimate for the truncation error 
${\cal T}_{n} - {\cal T}_\infty$ can be obtained
by summing the geometric series 
$\sum_{k=1}^\infty {\bar \rho}^k \, 
|{\cal T}_{n} - {\cal T}_{n-1}|$ where our best 
estimate for $\bar \rho$ is $\bar \rho \approx x_n$.
Therefore, we use the following convergence criterion to terminate 
the calculation of the CNC transforms:
%
\begin{equation}
\label{convcrit}
\frac{2}{x_n} \, \left[ \frac{1}{1-x_n} \,
\left| \frac{{\cal T}_{n} - {\cal T}_{n-1}}{{\cal T}_{n}} \right| \right] < 
{\tt acc}\,.
\end{equation}
%
Here, ${\tt acc}$ is the specified desired {\em relative} 
accuracy of the result. The factor $2/x_n$ in~(\ref{convcrit})
is a heuristic ``safeguard factor'' introduced with the 
notion of avoiding a premature termination of the calculation
of successive transforms in the problematic case of
two consecutive transforms accidentally assuming values
very close to each other. Such a situation may arise {\em before}
the asymptotic, geometric convergence sets in.
The term in square brackets in (\ref{convcrit}) represents 
the remainder estimate based on the geometric model $x_n \approx \bar \rho$
[see Eq.~(\ref{xx})].

Please note that {\tt lerchphi.c} relies on the header file 
{\tt float.h} which contains 
floating-point specifications for the minimum representable 
double precision number and machine epsilon on a system.  
This header file may not be present on all computer systems
on which {\tt lerchphi.c} is compiled.
If it is desired to code the relevant arithmetic constants
explicitly, then one should modify lines 50 and 51
of {\tt lerchphi.c}.

Finally, we would like to point out that the current
implementation of {\tt lerchphi} relies on only two 
algorithms (direct summation of the defining power series
and convergence acceleration), and that one cannot expect to 
obtain optimal performance in all parameter regions, let alone
analytic continuations for those cases where the power series
(\ref{lerch}) diverges. As regards the evaluation
of special functions with very large (excessive)
parameter values, it is known that asymptotic expansions can
provide optimal methods of evaluation. These are not implemented
in the current version of {\tt lerchphi}. 

Additionally, the use of the defining power series
(\ref{lerch}) entails {\em eo ipso} numerical problems,
when only double-precision arithmetic is used.
Consider the following two representative parameter combinations:
%
\begin{equation}
\label{cases}
\begin{array}{clll}
\mbox{(case 1:)} & z = 0.99999\,, & s = 2\,, & v = 10^3\,,\\[1ex]
%
\mbox{(case 2:)} & z = 0.0003\,, & s = 2\,, & v = -3.00000~00000~0001\,.
\end{array}
\end{equation}
%
When using the series (\ref{lerch}) in double precision, problems
result for both cases. 

{\bf Case 1.} The use of the CNCT implies
that terms of very large index of the input series have to be evaluated,
while avoiding the necessity to evaluate {\em all} terms until convergence
is reached. In case 1, the argument $z$ is very close to unity, 
but this should {\em a priori} not be a problem -- the CNCT is designed
to work with this problematic case. However, since $v$ is large,
the contribution of terms of large index as compared to the terms
of small index in (\ref{lerch}) becomes numerically more significant. 
This entails a loss of precision when trying to evaluate expressions
like $0.99999^L$ where $L$ is a very large integer. This expression can be
written as $\exp(L \, \ln 0.99999)$, and the numerical cancellations
encountered in the evaluation of $\ln 0.99999 \approx - 0.00001$ eventually
become significant when $L$ is excessively large. 
In the considered case, for a specified desired 
relative accuracy of $10^{-14}$, the double-precision version
{\tt lerchphi.c} yields a value of $9.59714\,89709\,9796 \times 10^{-4}$
which has to be compared to the true value of 
$9.59714\,89709\,9654\dots \times 10^{-4}$ obtained with arithmetic of
enhanced accuracy. The relative accuracy of the result obtained with
double-precision arithmetic is only $10^{-12}$ and thus smaller than
the specified accuracy goal.

{\bf Case 2.} The term with $n=3$ in Eq.~(\ref{lerch}) dominates
the sum, and its calculation entails considerable loss of numerical
significance in forming the difference $3-3.00000~00000~0001$,
corresponding to a loss of 14 significant decimals.
The double-precision version yields a value of $2.58\dots \times 10^{17}$,
which has to be compared to the value $2.70\dots \times 10^{17}$ obtained
with the extended-precision version. It is natural that after a loss
of 14 decimals at an intermediate stage of a calculation, the final
answer will provide not more than two significant decimals if
double-precision arithmetic (roughly 16 decimals) is used.

Consequently, both above cases find a solution in the 
extended-precision version of the code which is discussed
in the next section.

%
% Extended Precision
%
\section{Extended Precision}

The file {\tt lerchphimp.cc}, written in C++, contains a version
of {\tt lerchphi} that
uses multiprecision software libraries, e.g.~\cite{BaURL,BrURL}, 
to perform all floating-point arithmetic. 
The {\tt lerchphimp.cc}-code and the testing program {\tt testmp.cc} use the
multiprecision library~\cite{BaURL}; running the tests reveals 
that calculations in extended precision give correct 
results for the two problematic cases listed in
Eq.~(\ref{cases}). 

For the 8 representative example cases considered in the testing
routine, we have also carried out calculations with our own
Mathematica-based~\cite{Wo1988} implementation of the CNCT 
(file {\rm lerchphi.m})
and we compared with Mathematica's built-in {\tt LerchPhi}
routine. The speed comparison of {\tt lerchphimp.cc}
and Mathematica's built-in {\tt LerchPhi} is in favour of {\tt lerchphimp.cc}
for the 8 example calculations considered,
and final results are in mutual agreement.

Reports on further tests and comparisons are always welcome to the authors.

\begin{thebibliography}{10}

%\bibitem{JeAkMoSaSo2002}
%U.~D. Jentschura, S.~V. Aksenov, P.~J. Mohr, M.~A. Savageau, and G. Soff,
%  {\em Convergence Acceleration Techniques}, 
%  to be published in the Proceedings of
%  the ICCN--2002 conference (Computational Publications, 2002), e-print
%  math.NA/0202009.

\bibitem{JeBeSoAkSaMo2002} S.~V. Aksenov, M.~A. Savageau, U.~D. Jentschura,
  J. Becher, G. Soff, and P.~J. Mohr, 
  {\em Application of the Combined Nonlinear-Condensation Transformation to Problems in Statistical Analysis and Theoretical Physics}, 
Comput. Phys. Commun. {\bf 150}, 1 (2003), e-print math.NA/0207086. 
  
\bibitem{AkJeURL} 
S.~V.~Aksenov and U.~D.~Jentschura,
C and Mathematica Programs for Calculation of Lerch's Transcendent,\\
available at {\em http://aksenov.freeshell.org/} and\\
{\em http://tqd1.physik.uni-freiburg.de/\~{}ulj}.
  
\bibitem{Ba1953vol1}
H. Bateman, {\em Higher Transcendental Functions} (McGraw-Hill, New York, NY,
  1953), Vol.~1.
  
\bibitem{JeMoSoWe1999}
U.~D. Jentschura, P.~J. Mohr, G. Soff, and E.~J. Weniger, Comput. Phys. Commun.
  {\bf 116},  28  (1999).
  
\bibitem{vW1965}
A. van Wijngaarden, in {\em Cursus: Wetenschappelijk Rekenen B, Process
  Analyse} (Stichting Mathematisch Centrum, Amsterdam, 1965), pp.\ 51--60.

\bibitem{Si1979}
A. Sidi, Math. Comput. {\bf 33},  315  (1979).

\bibitem{We1989}
E.~J. Weniger, Comput. Phys. Rep. {\bf 10},  189  (1989).

\bibitem{Da1969}
J.~W. Daniel, Math. Comput. {\bf 23},  91  (1969).

\bibitem{BaURL} 
D.~H.~Bailey {\em et al.}, High precision arithmetic software, \\
available at {\em http://www.nersc.gov/\~{}dhbailey/mpdist/mpdist.html}.

\bibitem{BrURL} 
K.~M.~Briggs, Double-double floating point arithmetic, \\
available at {\em http://www.btexact.com/people/briggsk2/doubledouble.html}.

\bibitem{Wo1988}
S. Wolfram, {\em Mathematica-A System for Doing Mathematics by Computer}
  (Addison-Wesley, Reading, MA, 1988).

\end{thebibliography}

\end{document}
